\documentclass[10pt,twocolumn]{article}

%%%%%%%%%%%%%%%%%%%%%%%%%%%%%%%%%%%%%%%%%%%%%%%%%%
% Frequently Used Packages
\usepackage{amssymb,amsmath,amsfonts,amsthm}
\usepackage{times}
\usepackage{graphicx}
\usepackage{color}
\usepackage[pdftex,pagebackref,colorlinks,linkcolor=red,anchorcolor=blue,citecolor=green]{hyperref}
%%%%%%%%%%%%%%%%%%%%%%%%%%%%%%%%%%%%%%%%%%%%%%%%%%
% New Packages

%%%%%%%%%%%%%%%%%%%%%%%%%%%%%%%%%%%%%%%%%%%%%%%%%%


%%%%%%%%%%%%%%%%%%%%%%%%%%%%%%%%%%%%%%%%%%%%%%%%%%
% Format Definitions

%%%%%%%%%%%%%%%%%%%%%%%%%%%%%%%%%%%%%%%%%%%%%%%%%%


\title{How to Persuade A Rascal: Protocols for A Win-win Situation in Blockchain Systems}
\author{Input Authors Here}


\date{}

\begin{document}
\maketitle

\section{I}

%\cite{SP:ADMM14}

\subsection{I.1}

\section{II}

\section{Introduction}

For the first scenario, only timed-commitment transaction is enough. For the second scenario, we propose a new kind of transaction named as ``guaranteed transaction''. The essence of a guaranteed transaction is that the redeemer 

\section{Protocol I for a less sensitive client}

As stated above, suppose there exists a computing transaction $Cmpt$ which is mainly used to determine who is the winner. It is still waiting for the winner to redeem the coins but now there is a impasse between the players, because they do not utilize any protection mechanism before.



\bibliographystyle{plain}
\bibliography{mybibfile}
%\bibliography{abbrev0,crypto}

\end{document}
